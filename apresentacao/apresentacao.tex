% ||||||||||||||||||||||||||||||||||||||||||||||
% Arquivo de apresentação
% ||||||||||||||||||||||||||||||||||||||||||||||

\documentclass[aspectratio=169]{beamer}
\mode<presentation>
{
	\usetheme{Madrid}
    
    \definecolor{unisinos}{RGB}{32,64,154}
    \usecolortheme[named=unisinos]{structure}
    
	\usefonttheme{professionalfonts}
    \usebackgroundtemplate{\includegraphics[width=\paperwidth,height=0.978\paperheight]{template_unisinos}}

 	\beamertemplatenavigationsymbolsempty
 	\setbeamertemplate{footline}
	{
	  \leavevmode%
	  \hbox{%
	  \begin{beamercolorbox}[wd=.12\paperwidth,ht=2.25ex,dp=1ex,center]{author in head/foot}%
	    \usebeamerfont{author in head/foot}\insertshortauthor
	  \end{beamercolorbox}%
	  \begin{beamercolorbox}[wd=.88\paperwidth,ht=2.25ex,dp=1ex,center]{title in head/foot}%
	    \usebeamerfont{title in head/foot}\insertshorttitle\hspace*{3em}
	    \insertframenumber{} / \inserttotalframenumber\hspace*{1ex}
	  \end{beamercolorbox}}%
	  \vskip0pt%
	}
 	\setbeamertemplate{caption}[numbered]
 	\usepackage{caption}

 	\usepackage[brazil]{babel}		% Idioma do documento
	\usepackage{siunitx}
	\usepackage{ragged2e}
	\usepackage{float}
    \usepackage{indentfirst} }
    \usepackage[alf]{abntex2cite}		% Citações padrão ABNT
	\usepackage{color}			% Controle das cores
	\usepackage[T1]{fontenc}		% Selecao de codigos de fonte.
	\usepackage{graphicx}			% Inclusão de gráficos
	\usepackage[utf8]{inputenc}		% Codificacao do documento (conversão automática dos acentos)
	\usepackage{txfonts}			% Fontes virtuais
	\usepackage{tabularx}

\title[ Detecção e Diagnóstico Automático de Falhas em Motores Elétricos de Indução via Análise de Assinaturas na
Corrente Elétrica Utilizando ...]
{Detecção e Diagnóstico Automático de Falhas em Motores Elétricos de Indução via Análise de Assinaturas na
Corrente Elétrica Utilizando Técnicas de Clusterização}
\author[Piaia, G. A.]{
	{\fontsize{10}{8}\selectfont \textbf{Mestrando:} Guilherme Angelo Piaia} \\
	{\fontsize{10}{8}\selectfont \textbf{Orientador:} Prof. Dr. Rodrigo Marques de Figueiredo}
}
%\institute[UNISINOS]{UNISINOS - Universidade do Vale do Rio dos Sinos \\ Engenharia de Controle e Automação}

\date{\today}

\begin{document}

%%%%%%%%%%%%%%%%%%%%%%%%%%%%%%%%%%%

\begin{frame}
	\begin{minipage}{1\linewidth}
		\centering
		    \textbf{UNIVERSIDADE DO VALE DO RIO DOS SINOS - UNISINOS} \\ UNIDADE ACADÊMICA DE PESQUISA E PÓS-GRADUAÇÃO \\ PROGRAMA DE PÓS-GRADUAÇÃO EM ENGENHARIA ELÉTRICA \\ NÍVEL MESTRADO PROFISSIONAL
	\end{minipage}
	\titlepage
\end{frame}

%%%%%%%%%%%%%%%%%%%%%%%%%%%%%%%%%%%

\begin{frame}{Sumário}
	\tableofcontents
\end{frame}

%%%%%%%%%%%%%%%%%%%%%%%%%%%%%%%%%%%

\section{Introdução}
\begin{frame}{Introdução}
	\begin{itemize}
		\justifying
		\item Os motores elétricos de indução são amplamente empregados na indústria, pela sua confiabilidade e construção, o que o torna uma opção 
		barata e confiável \cite{Umans2003};
		\item A utilização de uma estratégia para uma manutenção ótima (preventiva), pode reduzir em até 65\% os custos de manutenção \cite{Wu2013};
		\item Monitoramento de limites de vibração, inteligência artificial e aprendizado de máquina podem ser
		utilizados para se criar uma técnica de manutenção preditiva;
		\item Este trabalho descreve a implementação de uma técnica de manutenção preditiva utilizando ferramentas
		modernas de processamento de sinais;
	\end{itemize}
\end{frame}

%%%%%%%%%%%%%%%%%%%%%%%%%%%%%%%%%%%

\section{Referencial Teórico}
\begin{frame}{Referencial Teórico - Motores Elétricos de Indução}
	\begin{columns}
    	\begin{column}{0.50\textwidth}
			\begin{figure}[HT]
				\begin{center}
					\captionsetup{justification=justified}
					\includegraphics[scale=.3]{../referencial/img/ind_motor_petruzella_p115.png}
					\caption{Motor elétrico de indução tipo gaiola de esquilo. \newline
					Fonte: \citeonline{Petruzella1911}.}
					\label{fig:ind_motor_petruzella_p115}
				\end{center}
			\end{figure}
     	\end{column}
		
		\begin{column}{0.5\textwidth}
			\begin{itemize}
				\item Motores elétricos de indução um tipo de máquinas elétricas, as quais convertem
				energia elétrica em mecânica;
			\end{itemize}
	 	\end{column}
	 \end{columns}
\end{frame}

%%%%%%%%%%%%%%%%%%%%%%%%%%%%%%%%%%%

\begin{frame}{Referencial Teórico - Motores Elétricos de Indução}
	\begin{columns}
    	\begin{column}{0.50\textwidth}
			\begin{figure}[HT]
				\begin{center}
					\captionsetup{justification=justified}
					\includegraphics[scale=.22]{../referencial/img/circuit_fitzgerald_p354.png}
					\caption{Circuito equivalente monofásico de um motor de indução polifásico. \newline
					Fonte: \citeonline{Umans2003}.} 
					\label{fig:circuit_fitzgerald_p354}
				\end{center}
			\end{figure}
			\begin{itemize}
				\item $\hat{V}_1$ = tensão no estator
				\item $\hat{E}_2$ = Força eletromotriz contrária gerada pelo fluxo no entreferro
			\end{itemize}
     	\end{column}
		
		\begin{column}{0.5\textwidth}
			\begin{itemize}
				\item $\hat{I}_1$ = corrente do estator
				\item $\hat{R}_1$ = resistência efetiva do estator
				\item $\hat{X}_1$ = reatância de vazamento do estator
				\item $R_c$ = resistência às perdas no núcleo
				\item $X_m$ = reatância magnetizante
				\item $\hat{I}_2$ = componente de corrente gerada pela carga
				\item $\hat{X}_2$ = reatância de vazamento do rotor no estator na frequência de escorregamento
				\item $R_2$ = resistência do rotor
				\item $\hat{I}_\varphi$ = componente de corrente \newline 
				excitada no estator
			\end{itemize}
	 	\end{column}
	 \end{columns}
\end{frame}

%%%%%%%%%%%%%%%%%%%%%%%%%%%%%%%%%%%

\begin{frame}{Referencial Teórico - Falhas em Motores Elétricos de Indução}
	\begin{columns}
    	\begin{column}{0.50\textwidth}
			\begin{figure}[HT]
				\begin{center}
					\captionsetup{justification=justified}
					\includegraphics[scale=.25]{../referencial/img/motor_system_rilski_p2.png}
					\caption{Visão geral de sistema com um motor elétrico. \newline
					Fonte: \citeonline{Gorbounov2018}.} 
					\label{fig:motor_system_rilski_p2}
				\end{center}
			\end{figure}
     	\end{column}
		
		\begin{column}{0.5\textwidth}
			\begin{figure}[HT]
				\begin{center}
					\captionsetup{justification=justified}
					\includegraphics[scale=.27]{../referencial/img/faults_rilski_p77.png}
					\caption{Árvore dos principais tipos de \newline
					falhas em motores elétricos. \newline
					Fonte: \citeonline{Gorbounov2018}.} 
					\label{fig:faults_rilski_p77}
				\end{center}
			\end{figure}
	 	\end{column}
	 \end{columns}
\end{frame}

%%%%%%%%%%%%%%%%%%%%%%%%%%%%%%%%%%%

\begin{frame}{Referencial Teórico - Falhas em Motores Elétricos de Indução}
	\begin{columns}
    	\begin{column}{0.50\textwidth}
			\begin{figure}[HT]
				\begin{center}
					\captionsetup{justification=justified}
					\includegraphics[scale=.25]{../referencial/img/misadraw_analog_p2.png}
					\caption{Ilustração com os tipos de desalinhamento. \newline
					Fonte: \citeonline{Sopcik2019}.} 
					\label{fig:misadraw_analog_p2}
				\end{center}
			\end{figure}
     	\end{column}
		
		\begin{column}{0.5\textwidth}
			\begin{figure}[HT]
				\begin{center}
					\captionsetup{justification=justified}
					\includegraphics[scale=.3]{../referencial/img/bearing_analog_p3.png}
					\caption{Imagens de rolamentos no \newline
					topo, e falhas na parte de baixo. \newline
					Fonte: \citeonline{Sopcik2019}.} 
					\label{fig:faults_rilski_p77}
				\end{center}
			\end{figure}
	 	\end{column}
	 \end{columns}
\end{frame}

%%%%%%%%%%%%%%%%%%%%%%%%%%%%%%%%%%%

\begin{frame}{Referencial Teórico - Análise de Vibração}
	\begin{columns}
    	\begin{column}{0.50\textwidth}
			\begin{figure}[HT]
				\begin{center}
					\includegraphics[scale=.25]{../referencial/img/fault_effect_randall_p54.png}
					\caption{Variação da carga para distinção da causa do efeito observado. \newline
					Fonte: \citeonline{Wu2013}.} 
					\label{fig:fault_effect_randall_p54}
				\end{center}
			\end{figure}
     	\end{column}
		
		\begin{column}{0.5\textwidth}
			\begin{itemize}
				\item  Com o aumento da carga e da vibração em conjunto, o efeito mecânico e elétrico pode ser 
				severo \citeonline{Wu2013};
				\item A vibração possui características específicas chamadas assinaturas;
				\item A análise dessas assinaturas pode levar ao diagnóstico da falha, ou a sua detecção;
			\end{itemize}
	 	\end{column}
	 \end{columns}
\end{frame}

%%%%%%%%%%%%%%%%%%%%%%%%%%%%%%%%%%%

\begin{frame}{Referencial Teórico - Análise de Vibração}
	\begin{columns}
    	\begin{column}{0.50\textwidth}
			\begin{figure}[HT]
				\begin{center}
					\includegraphics[scale=.28]{../referencial/img/iso10816-1_randall_p146.png}
					\caption{Tabela de valores eficaz máximos de velocidade para cada porte de máquina indicados pela norma ISO 10816-1. \newline
					Fonte: \citeonline{Wu2013}.} 
					\label{fig:iso10816-1_randall_p146}
				\end{center}
			\end{figure}
     	\end{column}
		
		\begin{column}{0.5\textwidth}
			\begin{itemize}
				\item A = bom estado
				\item B = aceitável
				\item C = apenas tolerável
				\item D = não permitido
				\item Classe I = pequenas máquinas (potência menor que $\SI{15}{\kilo\watt}$)
				\item Classe II = máquinas médias sem uma fundação especial (potência entre $\SI{15}{\kilo\watt}$ e $\SI{75}{\kilo\watt}$)
				\item Classe III = máquinas grandes sobre uma fundação rígida e pesada
				\item Classe IV = máquinas grandes \newline
				sobre uma fundação flexível (turbomáquinas) 
			\end{itemize}			
	 	\end{column}
	 \end{columns}
\end{frame}

%%%%%%%%%%%%%%%%%%%%%%%%%%%%%%%%%%%

\begin{frame}{Referencial Teórico - Análise de Vibração}
	\begin{columns}
    	\begin{column}{0.50\textwidth}
			\begin{figure}[HT]
				\begin{center}
					\includegraphics[scale=.25]{../referencial/img/misa_analog_p2.png}
					\caption{Indicação espectral de desalinhamento na velocidade. \newline
					Fonte: \citeonline{Sopcik2019}.} 
					\label{fig:misa_analog_p2}
				\end{center}
			\end{figure}
     	\end{column}
		
		\begin{column}{0.5\textwidth}
			\begin{figure}[HT]
				\begin{center}
					\includegraphics[scale=.31]{../referencial/img/imbalance_analog_p2.png}
					\caption{Indicação espectral de desbalanceamento na velocidade. \newline
					Fonte: \citeonline{Sopcik2019}.} 
					\label{fig:imbalance_analog_p2}
				\end{center}
			\end{figure}
	 	\end{column}
	 \end{columns}
\end{frame}

%%%%%%%%%%%%%%%%%%%%%%%%%%%%%%%%%%%

\begin{frame}{Referencial Teórico - Análise de Corrente Elétrica}
	\begin{columns}
    	\begin{column}{0.50\textwidth}
			\begin{figure}[HT]
				\begin{center}
					\includegraphics[scale=.18]{../referencial/img/fault_freq_randall_p55.png}
					\caption{Ilustração com as frequências associadas às falhas em rotores ou estatores de motores elétricos. \newline
					Fonte: \citeonline{Wu2013}.} 
					\label{fig:fault_freq_randall_p55}
				\end{center}
			\end{figure}
     	\end{column}
		
		\begin{column}{0.5\textwidth}
			\begin{figure}[HT]
				\begin{center}
					\includegraphics[scale=.3]{../referencial/img/current_benbouzid_p3.png}
					\caption{Indicação espectral de desbalanceamento na velocidade. \newline
					Fonte: \citeonline{El1999}.} 
					\label{fig:current_benbouzid_p3}
				\end{center}
			\end{figure}
	 	\end{column}
	 \end{columns}
\end{frame}

%%%%%%%%%%%%%%%%%%%%%%%%%%%%%%%%%%%

\begin{frame}{Referencial Teórico - Técnicas Modernas de Processamento de Sinais}
	\begin{columns}
    	\begin{column}{0.50\textwidth}
			\begin{figure}[HT]
				\begin{center}
					\includegraphics[scale=.27]{../referencial/img/monitoring_methods_rilski_p78.png}
					\caption{Árvore de métodos de monitoramento de falhas em motores elétricos. \newline
					Fonte: \citeonline{Gorbounov2018}.} 
					\label{fig:monitoring_methods_rilski_p78}
				\end{center}
			\end{figure}
     	\end{column}
		
		\begin{column}{0.5\textwidth}
			\begin{itemize}
				\item FFT (\textit{Fast Fourier Transform} - Transformada rápida de Fourier)
				\item CNN (\textit{Convolutional Neural network} - rede neural convolucional)
				\item ICA (\textit{Indepent Compoent Analysys} - análise de componentes independentes)
				\item t-SNE (\textit{t-Distributed Stochastic Neighbor Embedding} - Incorporação Estocástica de Vizinhos com
				Distribuição t)
				\item K-means 
			\end{itemize}			
	 	\end{column}
	 \end{columns}
\end{frame}

%%%%%%%%%%%%%%%%%%%%%%%%%%%%%%%%%%%

\begin{frame}{Referencial Teórico - Estado da Arte}
	\begin{columns}
    	\begin{column}{0.50\textwidth}
			\begin{figure}[HT]
				\begin{center}
					\includegraphics[scale=.25]{../referencial/img/cnn_ince_p2.png}
					\caption{Diagrama de um sistema que utiliza CNN. \newline
					Fonte: \citeonline{Ince2016}.} 
					\label{fig:cnn_ince_p2}
				\end{center}
			\end{figure}
     	\end{column}
		
		\begin{column}{0.5\textwidth}
			\begin{figure}[HT]
				\begin{center}
					\includegraphics[scale=.25]{../referencial/img/orbit_jeong_p3.png}
					\caption{Diagrama de uma técnica que \newline 
					utiliza classificação de orbitais com \newline
					CNN. \newline
					Fonte: \citeonline{Jeong2016}.} 
					\label{fig:orbit_jeong_p3}
				\end{center}
			\end{figure}	
	 	\end{column}
	 \end{columns}
\end{frame}

%%%%%%%%%%%%%%%%%%%%%%%%%%%%%%%%%%%

\begin{frame}{Referencial Teórico - Estado da Arte}
	\begin{columns}
    	\begin{column}{0.50\textwidth}
			\begin{figure}[HT]
				\begin{center}
					\includegraphics[scale=.27]{../referencial/img/proposta_zhang_p4.png}
					\caption{Técnica proposta por \citeonline{Zhang2018}. \newline
					Fonte: \citeonline{Zhang2018}.} 
					\label{fig:cnn_ince_p2}
				\end{center}
			\end{figure}
     	\end{column}
		
		\begin{column}{0.5\textwidth}
			\begin{figure}[HT]
				\begin{center}
					\includegraphics[scale=.25]{../referencial/img/results_zhang_p7.png}
					\caption{Resultados da técnica proposta \newline
					por \citeonline{Zhang2018}. \newline
					Fonte: \citeonline{Zhang2018}.} 
					\label{fig:orbit_jeong_p3}
				\end{center}
			\end{figure}	
	 	\end{column}
	 \end{columns}
\end{frame}

%%%%%%%%%%%%%%%%%%%%%%%%%%%%%%%%%%%

\begin{frame}{Referencial Teórico - Estado da Arte}
	\begin{figure}[HT]
		\begin{center}
			\includegraphics[scale=.35]{../referencial/img/ica_bracamonte_p4.png}
			\caption{Fluxograma de uma técnica que utiliza ICA. \newline
			Fonte: \citeonline{Garcia-Bracamonte2019}.} 
			\label{fig:ica_bracamonte_p4}
		\end{center}
	\end{figure}
\end{frame}

%%%%%%%%%%%%%%%%%%%%%%%%%%%%%%%%%%%

\section{Metodologia}
\begin{frame}{Metodologia - Configurações dos Experimentos}
	\begin{columns}
    	\begin{column}{0.50\textwidth}
			\begin{figure}[HT]
				\begin{center}
					\includegraphics[scale=.2]{../metodologia/img/real_plant.jpeg}
					\caption{Estrutura do simulador MFS ®. \newline
					Fonte: \citeonline{SpectraQuest2011}.}
					\label{fig:real_plant}
				\end{center}
			\end{figure}
     	\end{column}
		
		\begin{column}{0.5\textwidth}
			\begin{figure}[HT]
				\begin{center}
					\includegraphics[scale=.25]{../metodologia/img/lateral_desenho.png}
					\caption{Desenho simplificado da \newline 
					Planta. \newline 
					Fonte: Elaborado pelo Autor.}
					\label{fig:lateral_desenho}
				\end{center}
			\end{figure}
	 	\end{column}
	 \end{columns}
\end{frame}

%%%%%%%%%%%%%%%%%%%%%%%%%%%%%%%%%%%

\begin{frame}{Metodologia - Configurações dos Experimentos}
	\begin{table}[H]
		\label{tab:simulador}
		\centering%
		\begin{minipage}{\textwidth}
		  \begin{tabular*}{\textwidth}{cccc}
			\hline
			{Nome}                   & Falha                     & Freq. do motor [Hz] & Amostragem [kHz]\\ \hline
			\hline
			Bom                      &  Sem                      &      60             &    25  \\ 
			Misa15                   &  DA de 15 mils            &      60             &    10  \\
			Misa35\_10k\_20Hz        &  DA de 35 mils            &      20             &    25  \\
			Misa35\_10k\_20Hz\_unb   &  DA de 35 mils e DB       &      20             &    25  \\
			Misa35\_10k\_30Hz\_unb   &  DA de 35 mils e DB       &      30             &    25  \\
			Misa35\_10k\_40Hz        &  DA de 35 mils            &      40             &    25  \\
			Misa35\_10k\_5Hz         &  DA de 35 mils            &      5              &    25  \\
			Misa35\_10k\_80Hz        &  DA de 35 mils            &      80             &    25  \\
			Misa35\_8k               &  DA de 35 mils            &      20             &    20  \\ \hline
		  \end{tabular*}
		  \caption{Testes realizados. \newline
		  Fonte: Elaborado pelo Autor.} 
		\end{minipage}
	\end{table}
\end{frame}

%%%%%%%%%%%%%%%%%%%%%%%%%%%%%%%%%%%

\begin{frame}{Metodologia - T-SNE}
	\begin{figure}[HT]
		\begin{center}
			\includegraphics[scale=.42]{../metodologia/img/t-sne.png}
			\caption{Fluxograma da técnica que utiliza t-SNE. \newline
			Fonte: Elaborado pelo autor.}
			\label{fig:t-sne}
		\end{center}
	\end{figure}
\end{frame}

%%%%%%%%%%%%%%%%%%%%%%%%%%%%%%%%%%%

\begin{frame}{Metodologia - K-Means}
	\begin{figure}[HT]
		\begin{center}
			\includegraphics[scale=.4]{../metodologia/img/k-means.png}
			\caption{Fluxograma da técnica utilizando K-means. \newline
			Fonte: Elaborado pelo autor.}
			\label{fig:k-means}
		\end{center}
	\end{figure}
\end{frame}

%%%%%%%%%%%%%%%%%%%%%%%%%%%%%%%%%%%

\begin{frame}{Metodologia - ICA}
	\begin{figure}[HT]
		\begin{center}
			\includegraphics[scale=.38]{../metodologia/img/ica.png}
			\caption{Fluxograma da técnica que utiliza ICA. \newline
			Fonte: Elaborado pelo autor.}
			\label{fig:ica}
		\end{center}
	\end{figure}
\end{frame}

%%%%%%%%%%%%%%%%%%%%%%%%%%%%%%%%%%%

\section{Resultados Preliminares}
\begin{frame}{Resultados Preliminares - t-SNE}
	\begin{figure}[HT]
		\begin{center}
			\includegraphics[scale=.19]{../resultados/img/t-sne-1.png}
			\caption{Clusters resultantes utilizando a técnica t-SNE no domínio do tempo. \newline
			Fonte: Elaborado pelo Autor.}
			\label{fig:t-sne-1}
		\end{center}
	\end{figure}
\end{frame}

%%%%%%%%%%%%%%%%%%%%%%%%%%%%%%%%%%%

\begin{frame}{Resultados Preliminares - t-SNE}
	\begin{figure}[HT]
		\begin{center}
			\includegraphics[scale=.19]{../resultados/img/fft-t-sne-1.png}
			\caption{Clusters resultantes utilizando a técnica t-SNE no domínio da frequência. \newline
			Fonte: Elaborado pelo Autor.}
			\label{fig:fft-t-sne-1}
		\end{center}
	\end{figure}
\end{frame}

%%%%%%%%%%%%%%%%%%%%%%%%%%%%%%%%%%%

\begin{frame}{Resultados Preliminares - K-means}
	\begin{figure}[HT]
		\begin{center}
			\includegraphics[scale=.45]{../resultados/img/kmeans2.png}
			\caption{Clusters resultantes utilizando a técnica K-means. \newline
			Fonte: Elaborado pelo Autor.}
			\label{fig:kmeans2}
		\end{center}
	\end{figure}
\end{frame}

%%%%%%%%%%%%%%%%%%%%%%%%%%%%%%%%%%%

\begin{frame}{Resultados Preliminares - ICA}
	\begin{figure}[HT]
		\begin{center}
			\includegraphics[scale=.27]{../resultados/img/ica.png}
			\caption{Resultados da separação dos dados utilizando a técnica ICA. \newline
			Fonte: Elaborado pelo Autor.}
			\label{fig:ica}
		\end{center}
	\end{figure}
\end{frame}

%%%%%%%%%%%%%%%%%%%%%%%%%%%%%%%%%%%

\section{Cronograma}
\begin{frame}{Cronograma}
	\begin{columns}
    	\begin{column}{0.50\textwidth}
			\begin{figure}[HT]
				\begin{center}
					\includegraphics[scale=.18]{cronograma.png}
					\caption{Cronograma das atividades. \newline
					Fonte: Elaborado pelo Autor.} 
					\label{fig:crono}
				\end{center}
			\end{figure}
			\begin{itemize}
				\item 1 - Estudo e escolha do tema.
				\item 2 - Fundamentação teórica.
				\item 3 - Estudo da planta \textit{SpectraQuest Machinery Fault Simulator}.
			\end{itemize}		
     	\end{column}
		
		\begin{column}{0.5\textwidth}
			\begin{itemize}
				\item 4 - Aquisição dos dados das diferentes combinações de falhas de motores.
				\item 5 - Estudo e aplicação das técnicas definidas nas primeiras etapas.
				\item 6 - Escrita da Qualificação.
				\item 7 - Estudo de técnicas complementares.
				\item 8 - Apresentação.
				\item 9 - Criação de um software que embarque a solução.
				\item 10 - Testes, validação e \newline
				melhorias no software.
				\item 11 - Escrita da dissertação.
			\end{itemize}		
	 	\end{column}
	 \end{columns}
\end{frame}


%%%%%%%%%%%%%%%%%%%%%%%%%%%%%%%%%%%

\section{Considerações Finais}
	\begin{frame}{Considerações Finais}
    \begin{itemize}
			\justifying
			\item Diversidade de técnicas aplicáveis;
			\item Clara possibilidade de criação de clusters de acordo com cada tipo de falha;
			\item Possibilidade de isolamento das fontes de vibração;
			\item Cronograma bem distribuído e com tempo hábil;
	\end{itemize}
\end{frame}

%%%%%%%%%%%%%%%%%%%%%%%%%%%%%%%%%%%

\begin{frame}{Considerações Finais - Trabalhos futuros}
    \begin{itemize}
        \justifying
		\item Aplicação de uma CNN na classificação dos clusters; 
		\item Novas coletas de dados no MFS, explorando outras falhas e configurações de amostragem dos sinais;
		\item Criação de um software com interface gráfica para possibilitar o uso pelos responsáveis pela manutenção
		dos motores elétricos de indução;
		\end{itemize}
\end{frame}

%%%%%%%%%%%%%%%%%%%%%%%%%%%%%%%%%%%

\section{Referências Bibliográficas}
\begin{frame}[allowframebreaks]{Referências Bibliográficas}
	\bibliography{../referencial/references}
\end{frame}

%%%%%%%%%%%%%%%%%%%%%%%%%%%%%%%%%%%

\begin{frame}{Agradecimentos}
		\begin{center}
			{\Huge Obrigado pela Atenção!}
		\end{center}
\end{frame}

\end{document}