%% Modelo UNISINOS para Trabalhos de Conclusão de Curso (TCCs) baseado no abtex2-modelo-trabalho-academico.tex, v-1.9.5 laurocesar Copyright 2012-2015 by abnTeX2 group at http://www.abntex.net.br/ 
% ----------------------------------------------
% abnTeX2: Modelo de Trabalho Academico (tese de doutorado, dissertacao de mestrado e trabalhos monograficos em geral) em conformidade com ABNT NBR 14724:2011: Informacao e documentacao - Trabalhos academicos - Apresentacao
% ----------------------------------------------
% ==============================================
% ||||||||||||||||||||||||||||||||||||||||||||||
% ----------------------------------------------
\documentclass[
	% -- opções da classe memoir --
	12pt,				% tamanho da fonte
	openright,			% capítulos começam em pág ímpar (insere página vazia caso preciso)
	oneside,			% para impressão em verso e anverso. twoside Oposto a oneside
	a4paper,			% tamanho do papel. 
	% -- opções da classe abntex2 --
	chapter=TITLE,		% títulos de capítulos convertidos em letras maiúsculas
	%section=TITLE,		% títulos de seções convertidos em letras maiúsculas
	%subsection=TITLE,	% títulos de subseções convertidos em letras maiúsculas
	%subsubsection=TITLE,% títulos de subsubseções convertidos em letras maiúsculas
	% -- opções do pacote babel --
	english,			% idioma adicional para hifenização
%	french,				% idioma adicional para hifenização
%	spanish,			% idioma adicional para hifenização
	brazil				% o último idioma é o principal do documento
	]{abntex2} 
% ----------------------------------------------
% Pacotes de fontes... 
% ----------------------------------------------
\renewcommand{\ABNTEXchapterfont}{\fontfamily{ptm}\fontseries{sbc}\selectfont}
% \usepackage{lmodern}	% Usa a fonte Latin Modern
% \usepackage{fourier}	% Adobe utopia
% \usepackage{times}	% Usa a fonte Times, mas não muito boa
% \usepackage{palatino}	% Usa a fonte Palatino
% \usepackage{mathpazo}	% Usa a fonte Adobe Palatino

% \usepackage[scaled=.92]{helvet} % Usa a fonte Helvetica
% % Descomente a linha abaixo junto se for utilizar a fonte helvet
% \renewcommand{\familydefault}{\sfdefault}
\usepackage{mathptmx}
%\usepackage{tgtermes}
% Para entradas de capítulo com Times Bold usando um dos 2 pacotes acima
% \renewcommand{\ABNTEXchapterfont}{\rmfamily\bfseries}
% ----------------------------------------------
% Configuração das fontes
% ----------------------------------------------
% Algumas configurações de fontes para capitulos e seções tanto no texto quanto no sumário
\renewcommand{\ABNTEXchapterfont}{\bfseries}
\renewcommand{\ABNTEXchapterfontsize}{\Large}
\renewcommand{\ABNTEXpartfont}{\ABNTEXchapterfont}
\renewcommand{\ABNTEXpartfontsize}{\ABNTEXchapterfontsize}
\renewcommand{\cftpartfont}{\normalfont\bfseries}
\renewcommand{\ABNTEXsectionfont}{\bfseries}
\renewcommand{\ABNTEXsectionfontsize}{\large}
\renewcommand{\ABNTEXsubsectionfont}{\normalfont}
\renewcommand{\ABNTEXsubsectionfontsize}{\normalsize}
\renewcommand{\cftsubsectionfont}{\normalfont}
\renewcommand{\ABNTEXsubsubsectionfont}{\slshape}
\renewcommand{\cftsubsubsectionfont}{\normalfont\slshape}
\renewcommand{\ABNTEXsubsubsubsectionfont}{\bfseries}
% Para configurar mais níveis configure conforme utilizado acima e comente as duas linhas abaixo
\settocdepth{subsubsection} % configura sumário para apresentar subseções até o quarto nível
\setsecnumdepth{subsubsection} % configura para numerar subseções até o quarto nível. Subseções de quinto nível não conterão numeração.
\addto\captionsbrazil{\renewcommand{\listfigurename}{Lista de figuras}} % Altera nome da lista de ilustrações para lista de figuras
% \addto{\captionsbrazil}{\renewcommand{\bibname}{Referências Bibliográficas}}
% ----------------------------------------------
% Equações com numeração sequencial
% ----------------------------------------------
\usepackage{chngcntr}
\counterwithout{equation}{chapter}
% ----------------------------------------------
% Pacotes básicos 
% ----------------------------------------------
\usepackage[T1]{fontenc}	% Selecao de codigos de fonte.
\usepackage[utf8]{inputenc}	% Codificacao do documento (conversão automática dos acentos)
\usepackage{lastpage}		% Usado pela Ficha catalográfica
\usepackage{indentfirst} 	% Indenta o primeiro parágrafo de cada seção.
\usepackage{color, colortbl}% Controle das cores
\usepackage{float}
\usepackage{graphicx}		% Inclusão de gráficos
\usepackage{microtype} 		% para melhorias de justificação
\usepackage{array}
%\usepackage{gensymb}       % Símbolos
\usepackage{amsmath} 	%--------------------------%
\usepackage{hyperref} 	%--------------------------%
\usepackage{bibentry} 	% para inserir refs. bib. no meio do texto
% ----------------------------------------------
% Pacotes adicionais
% ----------------------------------------------
\usepackage{lipsum}	% para geração de dummy text
\usepackage[colorinlistoftodos, english]{todonotes}
  % exemplos de uso: 
  %\todo[inline, color=red!80]{texto}
\usepackage{verbatim}
\usepackage{soulutf8}
  % exemplos de uso: 
  %\hl{highlight} ou \st{strikeout} ou \ul{underline}
\usepackage{tabularx}
\usepackage{multirow}
%\usepackage{subfig}
\usepackage{pdfpages}
\usepackage{pgfplots}
  \pgfplotsset{compat=1.12}
% Para desenho de circuitos
\usepackage{tikz}
\usepackage[american]{circuitikz}
\usepackage{siunitx}
\sisetup{locale = FR}

% ---
% Formatação de código-fonte
% ---
\usepackage{listings}

% Altera o nome padrão do rótulo usado no comando \autoref{}
\renewcommand{\lstlistingname}{Código}

% Altera o rótulo a ser usando no elemento pré-textual "Lista de código"
\renewcommand{\lstlistlistingname}{Lista de códigos}

% Configura a ``Lista de Códigos'' conforme as regras da ABNT (para abnTeX2)
\begingroup\makeatletter
\let\newcounter\@gobble\let\setcounter\@gobbletwo
  \globaldefs\@ne \let\c@loldepth\@ne
  \newlistof{listings}{lol}{\lstlistlistingname}
  \newlistentry{lstlisting}{lol}{0}
\endgroup

\renewcommand{\cftlstlistingaftersnum}{\hfill--\hfill}

\let\oldlstlistoflistings\lstlistoflistings
\renewcommand{\lstlistoflistings}{%
   \begingroup%
   \let\oldnumberline\numberline%
   \renewcommand{\numberline}{\lstlistingname\space\oldnumberline}%
   \oldlstlistoflistings%
   \endgroup}



\usepackage{color}
\definecolor{dkgreen}{rgb}{0,0.6,0}
\definecolor{gray}{rgb}{0.5,0.5,0.5}
\definecolor{mauve}{rgb}{0.58,0,0.82}
\lstset{frame=tb,
  language=C,
  aboveskip=3mm,
  belowskip=3mm,
  showstringspaces=false,
  columns=flexible,
  basicstyle={\small\ttfamily},
  numbers=left,
  numberstyle=\tiny\color{gray},
  keywordstyle=\color{blue},
  commentstyle=\color{dkgreen},
  stringstyle=\color{mauve},
  breaklines=true,
  breakatwhitespace=true,
  tabsize=3
}     
% ----------------------------------------------
% Pacotes de citações
% ----------------------------------------------
\usepackage[brazilian,hyperpageref]{backref}	 % Paginas com as citações na bibl
\usepackage[alf, abnt-etal-text=it]{abntex2cite} % Citações padrão ABNT
% ==============================================
% CONFIGURAÇÕES DE PACOTES
% ==============================================
% ----------------------------------------------
% Configurações do pacote backref
% Usado sem a opção hyperpageref de backref
% ----------------------------------------------
\renewcommand{\backrefpagesname}{Citado na(s) página(s):~}
% Texto padrão antes do número das páginas
\renewcommand{\backref}{}
% Define os textos da citação
\renewcommand*{\backrefalt}[4]{
	\ifcase #1 %
		Nenhuma citação no texto.%
	\or
		Citado na página #2.%
	\else
		Citado #1 vezes nas páginas #2.%
	\fi}%
% ||||||||||||||||||||||||||||||||||||||||||||||
% Informações de dados para CAPA e FOLHA DE ROSTO
% ||||||||||||||||||||||||||||||||||||||||||||||
\titulo{DESENVOLVIMENTO DE UM CONTROLADOR PID COM AUTO SINTONIA USANDO REDES NEURAIS ARTIFICIAIS E REGRESSÃO NÃO-LINEAR ROBUSTA PARA O CONTROLE DE ATITUDE DE UM SIMULADOR DE SATÉLITES COM RODAS DE REAÇÃO}
\autor{GUILHERME ANGELO PIAIA}
\local{São Leopoldo, RS}
\data{2018}
\orientador{Prof. Dr. Rodrigo Marques de Figueiredo} 
%Nome completo do professor com sua titulação (Esp. ou MS. ou Dr.)
%\coorientador{Prof. Dr. }
\instituicao{%
  UNIVERSIDADE DO VALE DO RIO DOS SINOS - UNISINOS
  \par
  UNIDADE ACADÊMICA DE GRADUAÇÃO
  \par
  CURSO DE ENGENHARIA DE CONTROLE E AUTOMAÇÃO
  \par }
\tipotrabalho{Trabalho de conclusão de curso}
% O preambulo deve conter o tipo do trabalho, o objetivo, o nome da instituição e a área de concentração 
% Verifique a nomenclatura utilizada: Graduado, Bacharel ou Licenciado junto à Coordenação do seu Curso.
\preambulo{Trabalho de Conclusão de Curso
apresentado como requisito parcial para
obtenção do título de Bacharel em
Engenharia de Controle e Automação pelo curso de
Engenharia de Controle e Automação
da Universidade do Vale do Rio dos Sinos –
UNISINOS.}
% ----------------------------------------------
% Configurações de aparência do PDF final
% ----------------------------------------------
% alterando o aspecto da cor azul
\definecolor{blue}{RGB}{41,5,195}
% alterando o aspecto da cor cinza
\definecolor{gray}{RGB}{50,50,50}
% informações do PDF
\makeatletter
\hypersetup{
    %pagebackref=true,
    pdftitle={\imprimirtitulo}, 
    pdfauthor={\imprimirautor},
    pdfsubject={\imprimirpreambulo},
	pdfcreator={LaTeX - abnTeX2 - Overleaf},
	pdfkeywords={abnt}{latex}{abntex2}{trabalho acadêmico}{unisinos}{engenharia de controle e automação}{tcc},
    colorlinks=true, % false: boxed links; true: colored links
    linkcolor=black, % color of internal links
    citecolor=black, % color of links to bibliography
    filecolor=blue,  % color of file links
    urlcolor=gray,
    bookmarksdepth=4
}
\makeatother
% ----------------------------------------------
% Espaçamentos entre linhas e parágrafos 
% ----------------------------------------------
% O tamanho do parágrafo é dado por:
\setlength{\parindent}{1.3cm}
% Controle do espaçamento entre um parágrafo e outro:
\setlength{\parskip}{0.2cm}  % tente também \onelineskip
% ----------------------------------------------
% compila o indice
% ----------------------------------------------
\makeindex

% ----------------------------------------------
% Pacotes extras
% ----------------------------------------------
%\newcommand*\cuk{\´{C}}
 \addto{\captionsbrazil}{\renewcommand{\bibname}{Referências Bibliográficas}}
\newcommand{\cuk}{\'Cuk}
\usepackage{booktabs}
\usepackage{adjustbox}
\usepackage{graphicx}
\usepackage{placeins}
\usepackage{longtable}
\usepackage{caption}
% ----------------------------------------------