% ----------------------------------------------
% resumo em português
% ----------------------------------------------
\setlength{\absparsep}{18pt} % ajusta o espaçamento dos parágrafos do resumo
\begin{resumo}

Falhas em equipamentos industriais acarretam parada de produção, prejuízos e por consequência, perda de competitividade com aquelas que 
mitigam estes problemas. O desenvolvimento e implementação de soluções que busquem detectar e identificar falhas nestes equipamentos são, 
em sua maioria, indispensáveis para se minimizar os prejuízos de produção e os potenciais riscos à saúde das pessoas que trabalham nestes ambientes. 
Presente nestas máquinas, os motores elétricos de indução, que acoplados em um sistema mancalizado, criam o movimento necessário para a
realização da atividade objetivo da máquina. Estas partes que se movimentam, com o uso ou com alguma adversidade, em algum momento entrarão em 
colapso, evidenciando a falha, sendo então, necessário o monitoramento dos mesmos em tempo real para se prever e evitá-las.
O presente trabalho apresenta uma solução que integra \textit{software} e \textit{hardware} para resolver o problema antes citado, inclusive em tempo real e que 
aprende o comportamento do equipamento, sugerindo regiões de alerta e perigo, servindo de ferramenta para tomada de decisões.
Para o desenvolvimento, as grandezas físicas de vibração foram utilizadas, juntamente com técnicas de processamento de sinais e 
ML \textit{machine learning} (aprendizado de máquina). 
Após o desenvolvimento e implementação, a solução foi empregada em dois estudos de casos em empresas de diferentes ramos da indústria, que 
obtiveram resultados muito positivos, onde o sistema foi capaz de detectar com antecedência em um dos casos, e no outro, apesar do já estado 
avançado de desgaste do equipamento segundo a norma ISO 10816-1, informar um alerta.
  
	\vspace{\onelineskip}
	
	\noindent 
	\textbf{Palavras-chaves}: Análise de Vibração e temperatura. Detecção de Falhas. Monitoramento em tempo real. Motores elétricos de indução
\end{resumo}

% ----------------------------------------------
% resumo em inglês
% ----------------------------------------------
\begin{resumo}[Abstract]
 \begin{otherlanguage*}{english}
   This is the english abstract.
   \lipsum[7]
   \vspace{\onelineskip}
 
   \noindent 
   \textbf{Keywords}: Vibration. Fault Analysys.
 \end{otherlanguage*}
\end{resumo}
