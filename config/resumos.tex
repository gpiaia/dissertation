% ----------------------------------------------
% resumo em português
% ----------------------------------------------
\setlength{\absparsep}{18pt} % ajusta o espaçamento dos parágrafos do resumo
\begin{resumo}
    As técnicas de controle e sintonia de controladores foram se desenvolvendo e incorporando ferramentas desenvolvidas com o tempo. No presente panorama, o  uso de algoritmos que simulam o comportamento de organismo e métodos de otimização estão ganhando espaço e disputando com métodos clássicos em aplicações complexas e que exigem eficiência, elevada controlabilidade, como é o caso do controle de atitude de satélites. O presente trabalho apresenta um estudo sobre técnicas de sintonia de controladores PID e o desenvolvimento de uma, com o objetivo de controlar a dinâmica de um satélite. Portanto, nesse trabalho, propõe o desenvolvimento de um protótipo de satélite com rodas de reação e de um método automático de sintonia de controladores PID usando redes neurais artificiais e regressão não-linear robusta, afim de melhorar a dinâmica do mesmo. Por fim, são apresentados os resultados e analises do protótipo e do método automático de sintonia.
	\vspace{\onelineskip}
	\noindent 
	
	\textbf{Palavras-chaves}: Controladores PID, Redes Neurais Artificiais. Satélite. Controle de Atitude. Regressão Não-Linear Robusta. Rodas de Reação. 
\end{resumo}

% ----------------------------------------------
% resumo em inglês
% ----------------------------------------------
% \begin{resumo}[Abstract]
%  \begin{otherlanguage*}{english}
%    This is the english abstract.
%    \lipsum[7]
%    \vspace{\onelineskip}
 
%    \noindent 
%    \textbf{Keywords}: latex. abntex. text editoration.
%  \end{otherlanguage*}
% \end{resumo}
