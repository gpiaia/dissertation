% ----------------------------------------------
% inserir lista de ilustrações
% ----------------------------------------------
\pdfbookmark[0]{\listfigurename}{lof}
\listoffigures*
\cleardoublepage
% ----------------------------------------------
% inserir lista de listings
% ---------------------------------------------
\pdfbookmark[0]{\lstlistlistingname}{lol}
\begin{KeepFromToc}
\lstlistoflistings
\end{KeepFromToc}
\cleardoublepage
% ----------------------------------------------
% inserir lista de tabelas
% ----------------------------------------------
\pdfbookmark[0]{\listtablename}{lot}
\listoftables*
\cleardoublepage
% ----------------------------------------------
% inserir lista de abreviaturas e siglas
% ----------------------------------------------
% Deve conter a relação alfabética das siglas utilizadas no texto, seguidas das palavras ou das expressões escritas por extenso.
\begin{siglas}
%   \item[ABNT] Associação Brasileira de Normas Técnicas
%   \item[NBR] Normas Brasileiras de Regulação
\item[AD] \textit{Analógico-Digital}
\item[DC]  \textit{Direct Current} (Corrente contínua)
\item[ESC]  \textit{Electronic-Speed-Control} (Controlador eletrônico de velocidade)
\item[GMV]  \textit{General Minimum Variance} (Variância Mínima Geral)
\item[GPIO]  \textit{General Purpose Input/Output} (Entradas e Saídas de uso Geral)
\item[I2C]  \textit{Inter-Integrated Circuit} (Comunicação Entre circuitos Integrados)
\item[MMQ]  \textit{Método dos Mínimos Quadrados}
\item[PI] \textit{Proportionl and Integral} (Proporcional e Integral)
\item[PID] \textit{Proportionl, Integral and Derivative} (Proporcional, Integral e Derivativo)
\item[PWM]  \textit{Pulse Width Modulation} (Modulação por Largura de Pulso)
\item[RPM]  \textit{Rotações por Minuto}
\item[RMSE]  \textit{Root Mean Square Error} (Erro quadrático médio)
\item[RNA]  \textit{Rede Neural Artificial}
\item[Rpi]  \textit{Raspberry Pi Zero W}
\item[SSH]  \textit{Secure Shell} (Shell Seguro)
\item[UNISINOS]  \textit{Universidade do Vale do Rio dos Sinos}
\item[URSS]  \textit{União das Repúblicas Socialistas Soviéticas}
\item[ZOH]  \textit{Zero-Order-Hold} (Amostrador de Ordem Zero)

\end{siglas}
% ----------------------------------------------
% inserir lista de símbolos
% ----------------------------------------------
\begin{simbolos}
  \item[$F$] Força
  \item[$m$] Massa
  \item[$t$] Tempo
  \item[$dt$] Diferencial de tempo
  \item[$\vec{p}$] Vetor momento linear
  \item[$\vec{v}$] Vetor velocidade
  \item[$\vec{a}$] Vetor aceleração
  \item[$\vec{F}_i$] Força total sobre uma partícula
  \item[$\vec{f}_{ie}$] Força externa aplicada em uma partícula
  \item[$\vec{f}_{ij}$] Força interna aplicada por cada partícula j
  \item[$m_i$] Massa de uma partícula i
  \item[$\vec{a}_i$] Aceleração de uma partícula i
  \item[$\vec{F}_e$] Somatório das forças sobre um corpo
  \item[$\vec{F}_{ie}$] Somatório de todas as forças sobre as partículas
  \item[$\vec{r}_{com}$] Vetor posição 
  \item[$M_T$] Massa total de um corpo rígido
  \item[$x,y,z$] Coordenadas cartesianas
  \item[$I_{xx}, A, I_{yy}, B, I_{zz}$, C] Momentos de inércia de um corpo rígido simétricos 
  \item[$\omega,  \vec{\omega}$] Velocidade angular e vetor velocidade angular
  \item[$\dot{\vec{\omega}}$] Primeira derivada do vetor velocidade angular
  \item[$\omega_0$] Velocidade angular inicial
  \item[$\alpha$] Aceleração angular
  \item[$\vec{\tau}$] Vetor torque
  \item[$\vec{L}$] Vetor momento angular
  \item[$I$] Inércia
  \item[$r_{sat}$] Distância dos motores em relação ao centro de massa do satélite
  \item[$r_{e}$] Raio total da roda de reação
  \item[$r_{d}$] Raio do disco da roda de reação
  \item[$h_{r}$] Espessura da borda da roda de reação
  \item[$h_{d}$] Espessura do disco da roda de reação
  \item[$I_sat$] Inércia do satélite
  \item[$\psi, \theta, \phi$] Posição angular em relação aos eixos cartesianos
  \item[$\theta_e$] Valores desconhecidos de uma função não-linear
  \item[$\vec{L}_s$] Momento angular do satélite
  \item[$\vec{L}_{total}$] Soma do momento angular do satélite e das rodas de reação
  \item[$\vec{L}_{roda}$] Momento angular de uma roda de reação
  \item[$\vec{\omega}_{roda}$] Vetor velocidade angular de uma roda de reação
  \item[$\vec{L}_{\omega}$] Momento angular da roda de reação
  \item[$J_{\omega}, J_{\omega_2}$] Momento de inércia da roda de reação
  \item[$\vec{\psi}_{\omega}$] Variação instantânea da posição angular da roda de reação
  \item[$\tau_{\omega}$] Torque do conjunto motor/roda de reação
  \item[$\tau_{1}, \tau_{2}, \tau_{3}$] Torque do conjunto motor/roda de reação, um em cada eixo
  \item[$J_{\omega 1}$] Momento de inércia do rotor
  \item[$J$] Momento de inércia do rotor somado com a da roda de reação
  \item[$\omega_1, \omega_2, \omega_3$] Velocidade angular em relação aos eixos x, y e z, respectivamente
  \item[$\beta$] Posição
  \item[$\beta_0$] Posição inicial
  \item[$t_0$] Tempo inicial
  \item[$M_p$] Sobressinal
  \item[$t_0$] Tempo de atraso de transporte
  \item[$t_d$] Tempo de atraso de transporte
  \item[$t_r$] Tempo até sinal de referência
  \item[$t_p$] Tempo até o sobressinal
  \item[$t_s$] Tempo até o regime estacionário
  \item[$e,err$] Erro
  \item[$T_i$] Tempo integral
  \item[$T_d$] Tempo derivativo
  \item[$K, K_p$] Ganho Proporcional
  \item[$P$] Variável de processo
  \item[$K_i$] Ganho Integral
  \item[$K_d$] Ganho Derivativo
  \item[$\tau_m$] Torque do motor
  \item[$\beta_{com}$] Posição de referência
  \item[$G_e$] Função de transferência do controlador PID
  \item[$D$] Distúrbio
  \item[$G_p$] Função de transferência da planta
  \item[$\omega_{sp}$] Velocidade angular de referência
  \item[$y$] Saída do sistema
  \item[$u$] Sinal computado; Sinal do controlador
  \item[$T$] Período
  \item[$z$] Variável do plano z
  \item[$s$] Variável do plano s
  \item[$\hat{y}$] Sinal na saída de um ZOH; Saída estimada; Saída predita
  \item[$y*$] Trem de pulsos
  \item[$\hat{y}$] Saída predita
  \item[$K_u$] Ganho que leva o sistema à marginalidade
  \item[$d$] Amplitude de oscilação do relé
  \item[$\varepsilon$] Histerese do relé
  \item[$T_u$] É o período de oscilação
  \item[$H$] Ganho do filtro de Kalman
  \item[$\lambda_n$] Ganho do neurônio
  \item[$f_{rn}$] Função excitação do neurônio
  \item[$rn$] Saída de um neurônio
  \item[$p_n$] Enésimo dendrito
  \item[$a$] Sinal de saída do neurônio; Amplitude de oscilação do sinal de saída
  \item[$R_a$] Resistência de armadura
  \item[$L_a$] Indutância de armadura
  \item[$i_a$] Corrente de armadura
  \item[$e_a$] Tensão de armadura
  \item[$e_b$] Tensão induzida
  \item[$K_w$] Constante de velocidade contra-eletromotriz 
  \item[$K_t$] Constante de torque
  \item[$B$] Atrito viscoso
  \item[$V_a$] Tensão da fonte
\end{simbolos}