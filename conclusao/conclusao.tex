% ||||||||||||||||||||||||||||||||||||||||||||||
% Capitulo de Conclusão
% ||||||||||||||||||||||||||||||||||||||||||||||

\chapter[Conclusão]{Conclusão}

Na primeira parte do trabalho, foram apresentados conceitos básicos sobre motores elétricos de indução, suas falhas e métodos para se 
detectar e diagnosticá-las. Foram descritas análises que utilizam vibração para a detecção de falhas, com as
técnicas de processamento de sinais que elas utilizaram. Em resumo, o referencial descreve a possibilidade de se detectar
as falhas com o emprego de um ou mais métodos.

A metodologia, que está no capítulo 3, apresenta o simulador de falhas que foi utilizado, com as suas configurações e as
diferentes características dos testes realizados. Também foram descritas as topologias implementadas para se desenvolver a aplicação, passando 
pelo desenvolvimento de software, integração e nas configurações dos estudos de cados. 
Após a apresentação dos passos realizados, os resultados foram descritos e analisados.

Na análise dos resultados, que foi realizada no capítulo 4, ficou clara a aplicabilidade via a análise dos estudos de casos. O sistema foi
capaz de monitorar, armazenar e de alarmar um estado crítico, na iminência de uma falha em um dos estudos. Já no outro, que se trata do exaustor,
o sistema foi instalado tardiamente, cabendo apenas avaliar segundo a norma ISO 10816-1. Outra característica importante, é a utilização da
ferramenta de machine learning, que se mostrou muito perspicaz em criar linhas de alarme e perigo, quando o sistema é instalado em um dispositivo
que há pouco passou por manutenção, e no uso para grandezas com variância considerável. Já nos cenários em que ela não obteve bons resultados:
máquinas que já estão há tempo funcionando sem manutenção e grandezas com pouca variância, ficando a cargo do usuário avaliar se o uso do 
machine learning se aplica ou não. 

Trabalhos futuros a partir desse devem ser desenvolvidos, principalmente na adição de parâmetros de entrada, que utilizem a norma ISO 10816-1
por padrão. Tornar o software escalável e executável de forma remota, para garantir que os dados não serão comprometidos. Analisar estudos de 
casos de mais máquinas e motores, melhorando o modelo da aplicação. Por fim, trabalhar no desenvolvimento de técnicas de diagnóstico de falhas,
agregando valor ao sistema. 




