% ||||||||||||||||||||||||||||||||||||||||||||||
% Capitulo de Conclusão
% ||||||||||||||||||||||||||||||||||||||||||||||

\chapter[Considerações Finais]{considerações Finais}

Na primeira parte do trabalho, foram apresentados conceitos básicos sobre motores elétricos de indução, suas falhas e métodos para se 
detectar e diagnosticá-las. Foram descritas análises que utilizam vibração e a corrente para a detecção de falhas, com as
técnicas de processamento de sinais que elas utilizaram. Em resumo, o referencial descreve a possibilidade de se detectar e diagnosticar 
as falhas com o emprego de um ou mais métodos, independentemente se é via vibração ou corrente do estator.

A metodologia, que está no capítulo 3, apresenta o simulador de falhas que foi utilizado, com as suas configurações e as
diferentes características dos testes realizados. Também foram descritas as topologias implementadas para se processar os dados oriundos 
dos testes. Após a apresentação dos passos realizados, os resultados preliminares foram descritos e analisados.

Na análise dos resultados, que foi realizada no capítulo 4, ficou clara a possibilidade da criação de clusters de acordo com cada falha
que o motor pode apresentar, além de se isolar e detectar a fonte dos sinais, possibilitando além de detectar e diagnosticar a falha,
indicar a posição da falha em um sistema com diversos elementos mecânicos.

Já no capítulo 5, está o cronograma em forma de gráfico de Gantt que apresenta as principais atividades que já foram desenvolvidas
e as que ainda serão. Como podemos ver, as atividades estão distribuídas de acordo com a evolução do presente trabalho e em tempo hábil
para a execução das mesmas.

Trabalhos futuros a partir desse devem ser desenvolvidos, principalmente na adição de uma CNN após as aplicações de técnicas de aprendizado
de máquina, com o objetivo de classificar novos dados em tempo real. Coletar dados de falhas diferentes e em configurações diferentes, também 
será necessário pois, até agora foram abordados somente os problemas de desbalanceamento e desalinhamento, precisando cobrir as falhas por
rolamentos também. Por fim, a criação de um software com interface gráfica para a interação com os responsáveis pela manutenção dos
equipamentos. Esse software deverá contar com uma tela onde cada motor terá o seu estado de saúde disponibilizado, além do histórico
de falhas e níveis de vibração.





