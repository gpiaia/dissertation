% ||||||||||||||||||||||||||||||||||||||||||||||
% Capitulo de Metodologia
% ||||||||||||||||||||||||||||||||||||||||||||||

\chapter{Metodologia}

O presente capítulo descreve todas as etapas do desenvolvimento da proposta, indo desde a configuração do simulador de falhas MFS \textsuperscript \textregistered, da
empresa SpectraQuest \textsuperscript \textregistered, até as técnicas de processamento que foram utilizadas.

%++++++++++++++++++++++++++++++++++++++++++++++++++++++++++++++++
% 
%++++++++++++++++++++++++++++++++++++++++++++++++++++++++++++++++

\section{Configurações dos Experimentos em Laboratório}

Como dito anteriormente, fui utilizado um simulador de falhas em motores elétricos de indução e sistemas mancalizados, denominado
MFS \textsuperscript \textregistered (\textit{Machinery Fault Simulator} - simulador de falhas em máquinas), que pode ser visto na figura \ref{fig:real_plant}. O
simulador é composto por um motor elétrico de indução trifásico de 1 HP, o qual está conectado com um eixo via um acoplamento.
Esse eixo possui dois discos dourados furado, onde é possível se colocar cargas para criar um desbalanceamento no sistema.

\begin{figure}[H]
    \caption{Estrutura do simulador MFS \textsuperscript \textregistered.}
    \begin{center}
        \includegraphics[scale=.4]{metodologia/img/real_plant.jpeg}
    \end{center}
    \fonte{\citeonline{SpectraQuest2011}.} 
    \label{fig:real_plant}
\end{figure}

Para a realização dos testes, diversas combinações de falhas, frequências do inversor de frequências e frequências de amostragem foram
utilizadas, para se entender as dinâmicas das falhas diferentes condições de testes. A planta possui diversos acelerômetros e um sensor
de corrente, que está amostra uma das fases da alimentação. As falhas que foram inseridas nos testes foram a de desalinhamento dos
mancais e o desbalanceamento das cargas. Essas partes podem ser vistas na figura \ref{fig:lateral_desenho}.

\begin{figure}[H]
    \caption{Desenho simplificado da Planta.}
    \begin{center}
        \includegraphics[scale=.5]{metodologia/img/lateral_desenho.png}
    \end{center}
    \fonte{Elaborado pelo Autor.} 
    \label{fig:lateral_desenho}
\end{figure}

A tabela \ref{tab:simulador} contem as combinações de testes que foram realizados. Onde DA é desalinhamento e DB é desbalanceamento.
Cada um destes arquivos gerados possuem 9 colunas de dados, que correspondem aos 7 acelerômetros, 1 tacógrafo e um transformador de
corrente.

\begin{table}[H]
    \caption{Testes realizados.}
    \label{tab:simulador}
    \centering%
    \begin{minipage}{\textwidth}
      \begin{tabular*}{\textwidth}{cccc}
        \hline
        {Nome}                   & Falha                     & Freq. do motor [Hz] & Amostragem [kHz]\\ \hline
        \hline
        Bom                      &  Sem                      &      60             &    25  \\ 
        Misa15                   &  DA de 15 mils            &      60             &    10  \\
        Misa35\_10k\_20Hz        &  DA de 35 mils            &      20             &    25  \\
        Misa35\_10k\_20Hz\_unb   &  DA de 35 mils e DB       &      20             &    25  \\
        Misa35\_10k\_30Hz\_unb   &  DA de 35 mils e DB       &      30             &    25  \\
        Misa35\_10k\_40Hz        &  DA de 35 mils            &      40             &    25  \\
        Misa35\_10k\_5Hz         &  DA de 35 mils            &      5              &    25  \\
        Misa35\_10k\_80Hz        &  DA de 35 mils            &      80             &    25  \\
        Misa35\_8k               &  DA de 35 mils            &      20             &    20  \\ \hline
      \end{tabular*}
      \fonte{Elaborado pelo Autor.} 
    \end{minipage}
\end{table}
  
Após a apresentação de como foram realizados os testes no simulador, é possível aplicar as topologias propostas no presente trabalho nos
dados gerados. Esse assunto será abordado na próxima seção.


%++++++++++++++++++++++++++++++++++++++++++++++++++++++++++++++++
% 
%++++++++++++++++++++++++++++++++++++++++++++++++++++++++++++++++

\section{Configurações dos Experimentos em Campo}

Além do ambiente controlado, os conceitos foram aplicados em equipamentos reais e funcionais, que operam 24 horas por dia, 7 dias por semana. 
No decorrer do trabalho, foi possível aplicar em mais de 2 empresas a aplicação. A figura \ref{fig:exautor} representa um exaustor industrial,
um dispositivo que apresenta severas falhas em decorrência do acumulo de partículas que estão no ar. Uma falha num dispositivo desses, acarreta
em uma parada de produção, impactando diretamente no financeiro da empresa.

\begin{figure}[H]
    \caption{Fotografia externa de um exaustor industrical.}
    \begin{center}
        \includegraphics[scale=1]{metodologia/img/exaustor.png}
    \end{center}
    \fonte{Forncecida pela empresa.} 
    \label{fig:exautor}
\end{figure}

Em um deste dispositivo foi instalado um sensor industrial de vibração e temperatura, com o objetivo de coletar, processar e armazenar esses
dados, criando um dashboard com o estado de saúde deste motor. A figura \ref{fig:sensor_exaustor} apresenta o sensor instalado.

\begin{figure}[H]
    \caption{Fotografia externa do sensor instalado em um exaustor industrical.}
    \begin{center}
        \includegraphics[scale=1]{metodologia/img/sensor_exaustor.jpg}
    \end{center}
    \fonte{Forncecida pela empresa.} 
    \label{fig:sensor_exaustor}
\end{figure}


\begin{figure}[H]
    \caption{Diagrama de integração entre sensores e Software.}
    \begin{center}
        \includegraphics[scale=0.5]{metodologia/img/layout.PNG}
    \end{center}
    \fonte{Adaptado pelo autor.} 
    \label{fig:sensor_exaustor}
\end{figure}

\begin{figure}[H]
    \caption{Fluxo de acesso do Software.}
    \begin{center}
        \includegraphics[scale=0.8]{metodologia/img/fluxo_software.PNG}
    \end{center}
    \fonte{Elaborado pelo autor.} 
    \label{fig:sensor_exaustor}
\end{figure}


%++++++++++++++++++++++++++++++++++++++++++++++++++++++++++++++++
% 
%++++++++++++++++++++++++++++++++++++++++++++++++++++++++++++++++

\section{Processamento dos Sinais - Topologias Propostas}

Com o estudo dos conceitos básicos sobre os motores elétricos de indução, das técnicas de processamento de sinais e de como os dados foram
obtidos no simulador MFS \textsuperscript \textregistered, é possível aplicar essas técnicas de acordo com as topologias propostas no presente trabalho. As técnicas
bases que foram utilizadas são: t-SNE, K-means e ICA respectivamente.


%----------------------------------------------------------------
%  
%----------------------------------------------------------------

\subsection{T-SNE}

A ideia de se utilizar essa técnica surgiu do fato de que todas as simulações geraram uma quantidade muito grande de dados, e o interesse
de se visualizar todas essas informações em um único gráfico foi a estratégia usada para concentrar essas informações em um único 
gráfico, aparecendo o conceito de clusterização. Saber se era possível agrupar dados em clusters específicos de falhas. A figura 
\ref{fig:t-sne} apresenta a primeira rodada de testes com a técnica t-SNE. Já a segunda, foi fornecer exatamente os mesmo sinais, só que
processados antes pela FFT, passando do domínio do tempo para o domínio da frequência.

\begin{figure}[H]
    \caption{Fluxograma da técnica que utiliza t-SNE.}
    \begin{center}
        \includegraphics[scale=.65]{metodologia/img/t-sne.png}
    \end{center}
    \fonte{Elaborado pelo autor.} 
    \label{fig:t-sne}
\end{figure}

A implementação dessa técnica foi feita na linguagem de programação Python utilizando o pacote scikit-learn. Além disso, o software foi 
executado em uma máquina Linux de forma offline.


%----------------------------------------------------------------
%  
%----------------------------------------------------------------

\subsection{K-Means}

Seguindo a mesma linha da técnica anterior, a técnica K-means foi utilizada com o objetivo de clusterizar os dados em grupos de 4 
(K=4), pois existem dados de motor bom, com desalinhamento de 15 mils, 35 mils e desbalanceamento. A figura \ref{fig:k-means} apresenta
a topologia da implementação.


\begin{figure}[H]
    \caption{Fluxograma da técnica utilizando K-means.}
    \begin{center}
        \includegraphics[scale=.65]{metodologia/img/k-means.png}
    \end{center}
    \fonte{Elaborado pelo autor.} 
    \label{fig:k-means}
\end{figure}

Já na implementação dessa técnica, foi utilizada a linguagem de programação R, muito difundida dentro da área de aprendizado de máquina.
O algoritmo também foi executado em uma máquina Linux e de forma offline.


%----------------------------------------------------------------
%  
%----------------------------------------------------------------

\subsection{ICA}

Diferente das duas propostas anteriores, essa não aborda o conceito de clusterização, mas o de separar as fontes de sinais de vibração.
Para isso, utilizamos apenas três sinais amostrados da planta, sendo eles: a corrente, a aceleração na lateral do primeiro mancal e 
na lateral do motor. O objetivo dessa técnica é buscar na corrente os sinais de vibração que estão presentes na estrutura do simulador,
possibilitando além da detecção e do diagnóstico de uma falha, dizer em qual parte da máquina está o elemento que está em eminência de 
falhar. A figura \ref{fig:ica}

\begin{figure}[H]
    \caption{Fluxograma da técnica que utiliza ICA.}
    \begin{center}
        \includegraphics[scale=.65]{metodologia/img/ica.png}
    \end{center}
    \fonte{Elaborado pelo autor.} 
    \label{fig:ica}
\end{figure}


A implementação dessa técnica se deu no mesmo ambiente que a t-SNE.

Agora que todas todos os elementos das técnicas implementadas já foram apresentadas, os resultados preliminares podem ser descritos, os 
quais estão no próximo capítulo.