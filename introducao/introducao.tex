% ||||||||||||||||||||||||||||||||||||||||||||||
% Capitulo de Introdução
% ||||||||||||||||||||||||||||||||||||||||||||||

\chapter[Introdução]{Introdução}

Os motores elétricos de indução são amplamente empregados na indústria, pela sua confiabilidade e construção, o que torna uma opção 
barata e confiável \cite{Umans2003}. Mas com o tempo, esses equipamentos apresentam falhas decorrentes dos regimes de trabalho, esforços não 
projetados, ou somente pelo gestaste natural dos componentes. Nesses casos, a aplicação de uma estratégia de manutenção apropriada, que
monitore em tempo real e indique a manutenção em um tempo ótimo, pode diminuir em até 65\% os custos de manutenção \cite{Wu2013}. 

Essa técnica preventiva, que possibilita essa intervenção ótima, pode ser desenvolvida com diversas ferramentes, indo desde o monitoramento
de limites de vibração até a utilização de técnicas modernas de inteligência artificial e aprendizado de máquina. Indicar que a falha é
iminente e que a falha está se desenvolvendo, é imprescindível para um processo ser mais produtivo, pois, a manutenção pode ser programada
para um horário em que a máquina já estaria parada para outros fins, maximizando a produção, além do processo de manutenção ser mais rápido, 
já que os elementos não se danificaram por completo ainda.

Este trabalho tem como principal objetivo a implementação de uma técnica de manutenção preditiva, onde possibilite a detecção
de falhas via análise da vibração em motores elétricos de indução e sistemas mancalizados. Essa solução une técnicas de processamento de 
sinais e aprendizado de máquinas, com o objetivo de identificar limites de vibração para detecção de falhas de forma menos invasiva e mais 
acessível economicamente possível. Além disso, ao final do trabalho, os dados de dois estudos de casos são analisados.

O trabalho está organizado em capítulos, onde o presente capítulo é o primeiro. O segundo é o referencial teórico, que aborda conceitos
básicos sobre o trabalho, acrescentando o estado da arte ao final, onde é possível ver o estágio atual das tecnologias que se propõem 
resolver o mesmo problema. Seguindo, o terceiro capítulo é a metodologia, onde as simulações, desenvolvimento do software e os estudos de
casos são apresentadas. Já no quarto capítulo, os resultados são apresentados e com a análise dos mesmos. Por último, a 
conclusão, onde estão as conclusões e os próximos passos.




