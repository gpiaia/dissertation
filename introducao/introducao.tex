% ||||||||||||||||||||||||||||||||||||||||||||||
% Capitulo de Introdução
% ||||||||||||||||||||||||||||||||||||||||||||||

\chapter[Introdução]{Introdução}

Os motores elétricos de indução são amplamente empregados na indústria, pela sua confiabilidade e construção, o que o torna uma opção 
barata e confiável \cite{Umans2003}. Mas com o tempo, esses equipamentos apresentam falhas decorrentes dos regimes de trabalho, esforços não 
projetados, ou somente pelo gestaste natural dos componentes. Nesses casos, a aplicação de uma estratégia de manutenção apropriada, que
monitore em tempo real e indique a manutenção em um tempo ótimo, pode diminuir em até 65\% os custos de manutenção \cite{Wu2013}. 

Essa técnica preventiva, que possibilita essa intervenção ótima, pode ser desenvolvida com diversas ferramentes, indo desde o monitoramento
de limites de vibração até a utilização de técnicas modernas de inteligência artificial e aprendizado de máquina. Indicar que a falha é
iminente e diagnosticar em qual elemento a falha está se desenvolvendo, é imprescindível para um processo ser mais produtivo, pois 
a manutenção pode ser programada para um horário em que a máquina já estaria parada para outros fins, maximizando a produção, além do
processo de manutenção ser mais rápido, já sabendo onde atuar.

Este trabalho tem como principal objetivo a implementação de uma técnica de manutenção preditiva, onde possibilite a detecção e diagnóstico
de falhas via análise da corrente de motores elétricos de indução. Essa solução mescla técnicas de processamento de sinais, 
aprendizado de máquinas e redes neurais artificiais (RNAs), com o objetivo de identificar assinaturas na corrente elétrica que levem à
detecção e diagnóstico de falhas de formar menos invasiva e mais barata possível. Além disso, ao final do trabalho, comparar a acurácia 
do método desenvolvido no presente trabalho com as demais técnicas disponíveis.

O trabalho está organizado em capítulos, onde o presente capítulo é o primeiro. O segundo é o referencial teórico, que aborda conceitos
básicos sobre o trabalho, acrescentando o estado da arte ao final, onde é possível ver o estágio atual das tecnologias que se propõem 
resolver o mesmo problema. O terceiro capítulo é a metodologia, onde os materiais, configurações dos testes e as técnicas propostas são 
apresentadas. Já no quarto capítulo, os resultados preliminares são apresentados com a análise dos mesmos. Por último, as configurações 
finais, onde estão as conclusões até essa etapa do trabalho e os próximos passos.




